\documentclass[11pt,a4paper]{article}
\usepackage[utf8]{inputenc}
\usepackage[margin=1in]{geometry}
\usepackage{listings}
\usepackage{xcolor}
\usepackage{hyperref}
\usepackage{enumitem}
\usepackage{amsmath}

\lstdefinestyle{pythonstyle}{
    backgroundcolor=\color{white},   
    commentstyle=\color{green!60!black},
    keywordstyle=\color{blue},
    basicstyle=\ttfamily\small,
    breaklines=true,                 
    numbers=left,                    
    language=Python
}

\lstset{style=pythonstyle}

\title{\textbf{Day 2 Assignment: Circle Detector}\\
\large Hough Transform Implementation\\
\large Computer Vision Bootcamp}
\author{made with $\heartsuit$, by Aryan}
\date{Due: Next Session}

\begin{document}

\maketitle

\section{Objective}
Build a robust circle detection program using the Hough Circle Transform that can identify, analyze, and visualize circular objects in images.

\section{Background}

The Hough Circle Transform is a feature extraction technique used to detect circles in images. It's widely used in:
\begin{itemize}
    \item Quality control (detecting circular parts)
    \item Medical imaging (identifying cells, tumors)
    \item Traffic sign detection
    \item Coin counting and classification
    \item Industrial inspection
\end{itemize}

\subsection{How It Works}

The Hough Circle Transform uses a voting procedure in a 3D parameter space $(x, y, r)$ where:
\begin{itemize}
    \item $(x, y)$ is the circle center
    \item $r$ is the radius
\end{itemize}

For each edge point, the algorithm considers all possible circles that could pass through it.

\section{Algorithm Steps}

\subsection{Preprocessing}
\begin{enumerate}
    \item Convert image to grayscale
    \item Apply Gaussian blur to reduce noise
    \item (Optional) Enhance contrast with histogram equalization
\end{enumerate}

\subsection{Detection}
\begin{enumerate}
    \item Apply Hough Circle Transform with parameters:
    \begin{itemize}
        \item \texttt{dp}: Inverse ratio of accumulator resolution
        \item \texttt{minDist}: Minimum distance between circle centers
        \item \texttt{param1}: Upper threshold for Canny edge detector
        \item \texttt{param2}: Accumulator threshold (lower = more circles)
        \item \texttt{minRadius}, \texttt{maxRadius}: Size constraints
    \end{itemize}
\end{enumerate}

\subsection{Post-processing}
\begin{enumerate}
    \item Filter circles by confidence
    \item Remove duplicate detections
    \item Calculate statistics
\end{enumerate}

\section{Requirements}

\subsection{Functional Requirements}

Your program must:

\begin{enumerate}
    \item \textbf{Load and preprocess image}
    \begin{itemize}
        \item Accept image file path as input
        \item Convert to grayscale
        \item Apply appropriate blur
    \end{itemize}
    
    \item \textbf{Detect circles}
    \begin{itemize}
        \item Use \texttt{cv2.HoughCircles}
        \item Handle cases with no circles found
        \item Support parameter tuning
    \end{itemize}
    
    \item \textbf{Visualize results}
    \begin{itemize}
        \item Draw circle outlines (green)
        \item Mark circle centers (red)
        \item Label each circle with ID and radius
        \item Display original and annotated side-by-side
    \end{itemize}
    
    \item \textbf{Report statistics}
    \begin{itemize}
        \item Total number of circles detected
        \item Minimum, maximum, and average radius
        \item List all circles with coordinates and radii
    \end{itemize}
    
    \item \textbf{Save results}
    \begin{itemize}
        \item Save annotated image
        \item Export statistics to text file
    \end{itemize}
\end{enumerate}

\subsection{Code Quality}

\begin{itemize}
    \item Use functions (minimum 3 functions)
    \item Include docstrings for all functions
    \item Add parameter validation
    \item Handle errors gracefully
    \item Follow PEP 8 naming conventions
\end{itemize}

\subsection{Testing}

Test with at least 3 images:
\begin{enumerate}
    \item Coins or circular objects
    \item Image with overlapping circles
    \item Challenging image (noise, varying sizes)
\end{enumerate}

\section{Starter Code}

\begin{lstlisting}
import cv2
import numpy as np
import matplotlib.pyplot as plt

def preprocess_image(image_path):
    """
    Load and preprocess image for circle detection.
    
    Args:
        image_path: Path to input image
    
    Returns:
        tuple: (original_color, preprocessed_gray) or (None, None)
    """
    # TODO: Implement preprocessing
    pass


def detect_circles(gray_image, dp=1, minDist=50, param1=50, 
                   param2=30, minRadius=10, maxRadius=100):
    """
    Detect circles using Hough Circle Transform.
    
    Args:
        gray_image: Preprocessed grayscale image
        dp: Inverse accumulator resolution ratio
        minDist: Minimum distance between circle centers
        param1: Upper Canny threshold
        param2: Accumulator threshold
        minRadius: Minimum circle radius
        maxRadius: Maximum circle radius
    
    Returns:
        numpy array of circles (x, y, radius) or None
    """
    # TODO: Apply HoughCircles
    pass


def visualize_circles(image, circles, save_path=None):
    """
    Draw detected circles on image and display.
    
    Args:
        image: Original color image
        circles: Array of detected circles
        save_path: Optional path to save annotated image
    """
    # TODO: Draw circles and labels
    pass


def calculate_statistics(circles):
    """
    Calculate and display statistics about detected circles.
    
    Args:
        circles: Array of detected circles
    
    Returns:
        dict: Statistics dictionary
    """
    # TODO: Compute statistics
    pass


def main():
    """Main function."""
    # TODO: Implement main workflow
    pass


if __name__ == '__main__':
    main()
\end{lstlisting}

\section{Grading Rubric}

\begin{table}[h]
\centering
\begin{tabular}{|l|r|}
\hline
\textbf{Criteria} & \textbf{Points} \\
\hline
Preprocessing Implementation & 15 \\
Circle Detection (Hough Transform) & 25 \\
Visualization Quality & 20 \\
Statistics Calculation & 15 \\
Code Quality \& Organization & 15 \\
Testing (3 images) & 10 \\
\hline
\textbf{Total} & \textbf{100} \\
\hline
\end{tabular}
\end{table}

\section{Bonus Challenges}

\subsection{Bonus 1: Parameter Auto-Tuning (+15 points)}
Implement automatic parameter selection that adapts to different images.

\subsection{Bonus 2: Interactive GUI (+20 points)}
Create a GUI with sliders to adjust Hough parameters in real-time.

\subsection{Bonus 3: Video Processing (+15 points)}
Extend to detect circles in video frames and track them over time.

\subsection{Bonus 4: Size Classification (+10 points)}
Classify detected circles into size categories (small/medium/large) and color-code them.

\section{Tips}

\begin{itemize}
    \item Start with \texttt{param2=30}, adjust based on results
    \item Lower \texttt{param2} if missing circles
    \item Higher \texttt{param2} if detecting false circles
    \item Use \texttt{minDist} to prevent duplicate detections
    \item Blur is crucial - test kernel sizes 3, 5, 7, 9
    \item For overlapping circles, decrease \texttt{minDist}
\end{itemize}

\section{Submission}

Create \texttt{Day2\_Assignment\_YourName.zip} containing:
\begin{verbatim}
Day2_Assignment_YourName/
|-- circle_detector.py
|-- test_images/
|   |-- test1.jpg
|   |-- test2.jpg
|   +-- test3.jpg
|-- results/
|   |-- result1.jpg
|   |-- result2.jpg
|   |-- result3.jpg
|   +-- statistics.txt
+-- README.txt
\end{verbatim}

\vspace{1em}
\begin{center}
\textbf{Good luck with your circle detection!}
\end{center}

\end{document}
