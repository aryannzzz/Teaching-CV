\documentclass[11pt,a4paper]{article}
\usepackage[utf8]{inputenc}
\usepackage[margin=1in]{geometry}
\usepackage{listings}
\usepackage{xcolor}
\usepackage{hyperref}
\usepackage{enumitem}
\usepackage{graphicx}
\usepackage{amsmath}

\lstdefinestyle{pythonstyle}{
    backgroundcolor=\color{white},   
    commentstyle=\color{green!60!black},
    keywordstyle=\color{blue},
    numberstyle=\tiny\color{gray},
    stringstyle=\color{red!70!black},
    basicstyle=\ttfamily\small,
    breakatwhitespace=false,         
    breaklines=true,                 
    numbers=left,                    
    language=Python
}

\lstset{style=pythonstyle}

\title{\textbf{Day 1 Assignment: Pencil Sketch Effect}\\
\large Computer Vision Bootcamp}
\author{made with $\heartsuit$, by Aryan}
\date{Due: Next Session}

\begin{document}

\maketitle

\section{Objective}
Create a Python program that transforms any photograph into a realistic pencil sketch drawing using classical image processing techniques.

\section{Background}

The pencil sketch effect simulates traditional pencil drawing by:
\begin{itemize}
    \item Extracting edges and contours
    \item Creating grayscale tones that mimic pencil shading
    \item Using inversion and blending techniques
\end{itemize}

This technique is based on the "dodge and burn" method from photography, adapted for digital image processing.

\section{Algorithm}

Your program should implement the following steps:

\subsection{Step 1: Convert to Grayscale}
Convert the input color image to grayscale using OpenCV:
\begin{lstlisting}
gray = cv2.cvtColor(image, cv2.COLOR_BGR2GRAY)
\end{lstlisting}

\subsection{Step 2: Invert the Grayscale Image}
Create a negative by inverting pixel values:
\begin{equation}
    inverted(x, y) = 255 - gray(x, y)
\end{equation}

\subsection{Step 3: Apply Gaussian Blur}
Blur the inverted image with a large kernel (21x21):
\begin{lstlisting}
blurred = cv2.GaussianBlur(inverted, (21, 21), 0)
\end{lstlisting}

The large kernel size creates soft, diffused tones.

\subsection{Step 4: Invert the Blurred Image}
Invert again to prepare for blending:
\begin{equation}
    inverted\_blur(x, y) = 255 - blurred(x, y)
\end{equation}

\subsection{Step 5: Divide and Scale}
Create the sketch effect using division and scaling:
\begin{equation}
    sketch(x, y) = \min\left(255, \frac{gray(x, y)}{inverted\_blur(x, y)} \times 256\right)
\end{equation}

This step brightens highlights and emphasizes edges, creating the pencil sketch appearance.

\section{Requirements}

\subsection{Functional Requirements}
Your program must:
\begin{enumerate}
    \item Accept an image file path as input
    \item Implement the complete pencil sketch algorithm
    \item Display the original and sketch side-by-side
    \item Save the sketch output to a file
    \item Include error handling for:
    \begin{itemize}
        \item File not found
        \item Invalid image format
        \item Processing errors
    \end{itemize}
\end{enumerate}

\subsection{Code Quality Requirements}
\begin{itemize}
    \item Use functions to organize code (minimum 2 functions)
    \item Include docstrings for all functions
    \item Add inline comments for complex operations
    \item Follow Python naming conventions (PEP 8)
    \item Handle edge cases gracefully
\end{itemize}

\subsection{Testing Requirements}
Test your program with at least 3 different images:
\begin{enumerate}
    \item A portrait/face
    \item A landscape/scenery
    \item An object with distinct edges
\end{enumerate}

\section{Starter Code}

\begin{lstlisting}
import cv2
import numpy as np
import matplotlib.pyplot as plt

def pencil_sketch(image_path, blur_kernel=21):
    """
    Convert an image to pencil sketch effect.
    
    Args:
        image_path (str): Path to input image
        blur_kernel (int): Gaussian blur kernel size (must be odd)
    
    Returns:
        tuple: (original_rgb, sketch) or (None, None) if error
    """
    # TODO: Implement the algorithm
    # Step 1: Load image
    
    # Step 2: Convert to grayscale
    
    # Step 3: Invert grayscale
    
    # Step 4: Apply Gaussian blur
    
    # Step 5: Invert blurred image
    
    # Step 6: Divide and scale
    
    pass


def display_result(original, sketch, save_path=None):
    """
    Display original and sketch side-by-side.
    
    Args:
        original: Original image (RGB)
        sketch: Sketch image (grayscale)
        save_path: Optional path to save the sketch
    """
    # TODO: Create matplotlib figure with 1 row, 2 columns
    # Display original on left, sketch on right
    # Add titles and remove axes
    # If save_path provided, save the sketch
    
    pass


def main():
    """Main function to run the pencil sketch converter."""
    # TODO: Get image path from user or command line
    # Call pencil_sketch function
    # Call display_result function
    # Handle any errors
    
    pass


if __name__ == '__main__':
    main()
\end{lstlisting}

\section{Submission Guidelines}

\subsection{What to Submit}
\begin{enumerate}
    \item Python script (\texttt{pencil\_sketch.py})
    \item 3 test images (original)
    \item 3 output sketches
    \item README file with:
    \begin{itemize}
        \item How to run your program
        \item Any dependencies
        \item Observations about results
        \item Any challenges faced
    \end{itemize}
\end{enumerate}

\subsection{Submission Format}
Create a ZIP file named \texttt{Day1\_Assignment\_YourName.zip} containing:
\begin{verbatim}
Day1_Assignment_YourName/
|-- pencil_sketch.py
|-- test_images/
|   |-- test1.jpg
|   |-- test2.jpg
|   +-- test3.jpg
|-- output_sketches/
|   |-- sketch1.jpg
|   |-- sketch2.jpg
|   +-- sketch3.jpg
+-- README.txt
\end{verbatim}

\section{Grading Rubric}

\begin{table}[h]
\centering
\begin{tabular}{|l|r|}
\hline
\textbf{Criteria} & \textbf{Points} \\
\hline
Algorithm Implementation & 30 \\
Code Quality \& Organization & 20 \\
Error Handling & 10 \\
Display \& Visualization & 15 \\
Testing (3 images) & 15 \\
Documentation & 10 \\
\hline
\textbf{Total} & \textbf{100} \\
\hline
\end{tabular}
\end{table}

\section{Bonus Challenges}

Implement any of these for extra credit:

\subsection{Bonus 1: Adjustable Blur Parameter (+10 points)}
Add command-line argument or user input to adjust blur kernel size dynamically.

\subsection{Bonus 2: Color Pencil Sketch (+15 points)}
Create a colored sketch version:
\begin{itemize}
    \item Convert to HSV color space
    \item Apply sketch algorithm to Value channel only
    \item Combine with Hue and Saturation
    \item Add slight desaturation for realistic effect
\end{itemize}

\subsection{Bonus 3: Video Processing (+20 points)}
Extend your program to process video files frame-by-frame:
\begin{itemize}
    \item Read video using \texttt{cv2.VideoCapture}
    \item Apply sketch effect to each frame
    \item Write output video using \texttt{cv2.VideoWriter}
    \item Display progress bar
\end{itemize}

\subsection{Bonus 4: Interactive GUI (+25 points)}
Create a simple GUI using Tkinter or similar:
\begin{itemize}
    \item File selection dialog
    \item Sliders to adjust blur kernel size
    \item Real-time preview
    \item Save button
\end{itemize}

\section{Tips for Success}

\begin{itemize}
    \item \textbf{Start simple:} Get the basic algorithm working first before adding features
    \item \textbf{Test incrementally:} Display intermediate results after each step
    \item \textbf{Handle division by zero:} Add small epsilon when dividing
    \item \textbf{Experiment with kernel size:} Try different values (15, 21, 25, 31)
    \item \textbf{Check data types:} Ensure proper uint8 conversion for display
    \item \textbf{Use git:} Version control helps track your progress
\end{itemize}

\section{Common Pitfalls}

\begin{enumerate}
    \item \textbf{Division by zero:} Add \texttt{+ 1e-6} to denominator
    \item \textbf{Overflow errors:} Use \texttt{np.clip()} to constrain values
    \item \textbf{Wrong data types:} Convert to \texttt{uint8} for display
    \item \textbf{BGR vs RGB:} Remember OpenCV uses BGR
    \item \textbf{Odd kernel sizes:} Gaussian blur requires odd kernel sizes
\end{enumerate}

\section{Resources}

\begin{itemize}
    \item OpenCV Documentation: \url{https://docs.opencv.org}
    \item NumPy Documentation: \url{https://numpy.org/doc}
    \item Matplotlib Gallery: \url{https://matplotlib.org/stable/gallery}
\end{itemize}

\vspace{1em}
\begin{center}
\textbf{Good luck! We look forward to seeing your creative sketches!}
\end{center}

\end{document}
